\documentclass[lang=cn,newtx,10pt,scheme=chinese]{elegantbook}

\title{微积分习题册参考答案}
\subtitle{大学数学习题册第三版}
\author{$\mathcal{Y}i\mathcal{S}hao$}
\bioinfo{单位}{$\mathcal{SCU}$\&Shishi Experimental School of Chengdu Eastern New Area}
\date{\today}
%\version{4.5}


\extrainfo{海纳百川,有容乃大}

\setcounter{tocdepth}{1}

\logo{logo_scu.png}
\cover{cover1.jpg}

% 本文档命令
\usepackage{array}

\usepackage{extarrows}
\newcommand{\ccr}[1]{\makecell{{\color{#1}\rule{1cm}{1cm}}}}
\usepackage{amssymb}
% 修改标题页的橙色带
%\definecolor{customcolor}{RGB}{255,170,170}
\definecolor{customcolor}{RGB}{244,105,102}
\colorlet{coverlinecolor}{customcolor}
\usepackage{cprotect}

\addbibresource[location=local]{reference.bib} % 参考文献,不要删除

%\input{xecjkfonts},CJKtextspaces
\setCJKmainfont{楷体}%{STKaiti}   % STFangsong 设置缺省中文字体
\setmainfont{Times New Roman} % 英文衬线字体
\setmonofont{Times New Roman} % 英文等宽字体
\setsansfont{Times New Roman} % 英文无衬线字体


\begin{document}

\maketitle
\frontmatter
\chapter*{前言}
该参考答案讲义采用了\href{https://elegantlatex.org/}{Elegant\LaTeX}模板,官网:\href{https://elegantlatex.org/}{https://elegantlatex.org/},在此向开发者表示诚挚的谢意。\\

该参考答案仅供内部交流使用,不建议随意发布到网上。(水平有限)\\


目前更新进度主要适用于微积分2下册的同学哈!有任何疑问,请联系左博团队的艺少.\\

邮箱:571136772@qq.com

\tableofcontents

\mainmatter





\chapter{定积分的定义及性质}
~\\
一、利用定积分的定义计算下列定积分.\\

\begin{enumerate}
\item $\displaystyle\int_a^b x \mathrm{~d} x(a<b)$;\\
\begin{solution}\\
	$
	\begin{aligned}
		\mathcal{I}& =\lim _{n \rightarrow \infty} \frac{b-a}{n} \cdot \sum_{i=1}^n\left(a+\frac{b-a}{n} \cdot i\right) =\lim _{n \rightarrow \infty} \frac{b-a}{n} \cdot\left(n a+\frac{b-a}{n} \cdot \frac{n(n+1)}{2}\right) \\
		& =\lim _{n \rightarrow \infty} a(b-a)+\frac{(b-a)^2}{n^2} \cdot \frac{n(n+1)}{2} \\
		& =a(b-a)+\frac{(b-a)^2}{2} =\frac{1}{2}\left(b^2-a^2\right).
	\end{aligned}
	$
\end{solution}

\item $\displaystyle\int_0^1 \mathrm{e}^x \mathrm{~d} x$;
\begin{solution}\\
	$
	\begin{aligned}
		\mathcal{I}& =\lim _{n \rightarrow \infty} \frac{1}{n} \cdot \sum_{i=1}^n e^{\frac{i}{n}}=\lim _{n \rightarrow \infty} \frac{e^{\frac{1}{n}}(1-e)}{n \cdot\left(1-e^{\frac{1}{n}}\right)}.\\
		&\text{对分母有:} \lim _{n \rightarrow \infty} n \cdot\left(1-e^{\frac{1}{n}}\right)\xlongequal[]{\text{令}x=\frac{1}{n}}\lim _{x \rightarrow 0} \frac{1}{x} \cdot\left(1-e^x\right)=-1.\\
		&\text{因此:}\int_0^1 \mathrm{e}^x \mathrm{~d} x=e-1.
	\end{aligned}
	$
\end{solution}
\end{enumerate}
~\\
二、利用定积分的几何意义, 求下列定积分.\\

\begin{enumerate}
	\item $\displaystyle\int_{-a}^a \sqrt{a^2-x^2} \mathrm{~d} x$;
	\begin{solution}
		相当于是一个半径为$a$的半圆的面积,答案为$\dfrac{\pi}{2}a^2$.
	\end{solution}
~\\
\item $\displaystyle\int_{-1}^3 x \mathrm{~d} x$;
\begin{solution}
作图可知答案为4.
\end{solution}
~\\
\item $ \displaystyle\int_{-\frac{\pi}{2}}^{\frac{\pi}{2}} \sin x \mathrm{~d} x$;
\begin{solution}
	画图可知答案为0.
\end{solution}
~\\

\item $\displaystyle\int_a^b(k x+m) \mathrm{d} x(0 \leqslant a<b)$;
\begin{solution}
	画图可知梯形面积为: $\dfrac{k}{2}(b^2-a^2)+m(b-a)\Rightarrow \mathcal{I}=\dfrac{k}{2}(b^2-a^2)+m(b-a)$.
\end{solution}
~\\

\item 设函数 $f(x)$ 在 $[0,+\infty)$ 上连续且单调增加, $f(0)=0, x=g(y)$ 是其反函数, 试用定积 分的几何意义说明下式成立:
$$
\displaystyle\int_0^a f(x) \mathrm{d} x+\int_0^b g(x) \mathrm{d} x \geqslant a b(a>0, b>0) \text {. }\qquad\textcolor{purple}{[Young \text{不等式}]}
$$

\begin{proof}
	 \ding{172} 先证当 $b=f(a)$ 时等号成立.\\
	将区间 $[0, a]$ 作划分: $0=x_0<x_1<\cdots<x_{n-1}<x_n=a$ ,记
	$y_i=f\left(x_i\right)(i=0,1,2, \cdots, n)$, 则\\ $0=y_0<y_1<\cdots<y_{n-1}<y_n=b$, 再记
	$\Delta x_i=x_i-x_{i-1}, \Delta y_i=y_i-y_{i-1}$, 于是
	$$
	\begin{aligned}
		& \sum_{i=1}^n f\left(x_{i-1}\right) \Delta x_i+\sum_{i=1}^n f^{-1}\left(y_i\right) \Delta y_i=\sum_{i=1}^n y_{i-1}\left(x_i-x_{i-1}\right)+\sum_{i=1}^n x_i\left(y_i-y_{i-1}\right)  =x_n y_n-x_0 y_0=a b,
	\end{aligned}
	$$
	记 $\lambda=\max\limits _{1 \leq i \leq n}\left\{\Delta x_i\right\}$, 当 $\lambda \rightarrow 0$ 时, $\sum_{i=1}^n f\left(x_{i-1}\right) \Delta x_i+\sum_{i=1}^n f^{-1}\left(y_i\right) \Delta y_i$ 的极限为\\
	$$
	\int_0^a f(x) d x+\int_0^b f^{-1}(y) d y,
	$$
	这就证明了当 $b=f(a)$ 时, $$\int_0^a f(x) d x+\int_0^b f^{-1}(y) d y=a b,$$ 
	\ding{173} 在一般情况下, 设 $$F(a)=\int_0^a f(x) d x+\int_0^b f^{-1}(y) d y-a b.$$则 $F^{\prime}(a)=f(a)-b$, 记 $f(T)=b$, 可知当 $0<a<T$ 时, $F(a)$ 单调减少, 当 $a>T$ 时, $F(a)$ 单调增加, 所以 $F(a)$ 在 $a=T$ 处取到最小值。由上面的讨论, 可知最小值 $F(T)=0$, 从而 $F(a) \geq 0$, 这就是所要证明的.
	 $\hfill\square$
\end{proof}
{\color{purple}作图方法见习题课.}
\end{enumerate}



~\\
三、用定积分表示下列数列极限.\\
\begin{enumerate}
	\item $\lim\limits _{n \rightarrow \infty} \dfrac{1}{n}\left(\sqrt[3]{1+\dfrac{1}{n}}+\sqrt[3]{1+\dfrac{2}{n}}+\cdots+\sqrt[3]{1+\dfrac{n}{n}}\right)$;
\begin{solution}
	原式 $=\lim\limits _{n \rightarrow \infty} \dfrac{1}{n} \cdot \displaystyle\sum_{i=1}^n \sqrt[3]{1+\frac{i}{n}}=\int_0^1 \sqrt[3]{1+x} d x.
	$
\end{solution}
	\item $\lim\limits _{n \rightarrow \infty} \dfrac{h}{n}\left\{\sin a+\sin \left(a+\dfrac{h}{n}\right)+\sin \left(a+\dfrac{2 h}{n}\right)+\cdots+\sin \left[a+\dfrac{(n-1) h}{n}\right]\right\}$.
	\begin{solution}\\
		$
		\begin{aligned}
			\text{原式 }& =\lim _{a \rightarrow \infty} \frac{h}{n}\left\{\sin a+\sin \left(a+\frac{h}{n}\right)+\cdots+\sin \left(a+\frac{n}{n} h\right)\right\} \\
			& =\lim _{n \rightarrow \infty} \frac{(a+h)-a}{n} \cdot \sum_{i=1}^n \sin \left(a+\frac{i}{n} \cdot h\right) \\
			& =\int_a^{a+h} \sin x d x.
		\end{aligned}
		$
	\end{solution}
\end{enumerate}

~\\
四、试用定积分的几何意义解释以下性质.\\
\begin{enumerate}
	\item 若 $f(x)$ 是奇函数,则 $\displaystyle\int_{-a}^a f(x) \mathrm{d} x=0$;
	\item 若 $f(x)$ 是偶函数, 则 $\displaystyle\int_{-a}^a f(x) \mathrm{d} x=2 \displaystyle\int_0^a f(x) \mathrm{d} x$;
	\item  $\displaystyle\int_a^b f(x) \mathrm{d} x=\int_a^b f(a+b-x) \mathrm{d} x{\color{purple} =\dfrac{1}{2}[\displaystyle\int_a^b f(x) \mathrm{d} x+\int_a^b f(a+b-x) \mathrm{d} x]}$. \quad\textcolor{purple}{[区间再现公式]}

\end{enumerate}
~\\



五、利用定积分的定义证明以下性质.\\

$\displaystyle\int_a^b k f(x) \mathrm{d} x=k \int_a^b f(x) \mathrm{d} x(k \text { 是常数). } $\\

\begin{proof}
	$\displaystyle \int_a^b k \cdot f(x) d x=\lim _{n \rightarrow \infty} \frac{b-a}{n} \cdot \sum_{i=1}^n k \cdot f\left(a+\frac{b-a}{n} \cdot i\right) 
	=k \cdot \lim _{n \rightarrow \infty} \frac{b-a}{n} \cdot \sum_{i=1}^n f\left(a+\frac{b-a}{n} i\right)=k \cdot \displaystyle\int_a^b f(x) d x.$$\hfill\square$
\end{proof}

~\\
六、设 $D(x)$ 是狄利克雷函数:\\
$$
D(x)=\left\{\begin{array}{l}
	1, \text { 当 } x \text { 是有理数, } \\
	0, \text { 当 } x \text { 是无理数. }
\end{array}\right.
$$

问:定积分 $\displaystyle\int_a^b D(x) \mathrm{d} x(a<b)$ 是否存在?为什么?\\
\begin{solution}
	先假设存在. $I=\displaystyle\int_a^b D(x) d x=\lim _{n \rightarrow \infty} \frac{b-a}{n} \cdot \sum_{i=1}^n D\left(x_i\right), x_i=a+\frac{b-a}{n} \cdot i.
	$\\
\ding{172} 取 $x_i \displaystyle\in Q . \quad D\left(x_i\right)=1 . \quad I=\lim _{n \rightarrow \infty} \frac{b-a}{n} \cdot n=b-a>0$.\\
\ding{173} 取 $x_i \displaystyle\notin Q . \quad D\left(x_i\right)=0 . \quad I=\lim _{n \rightarrow \infty} \frac{b-a}{n} \cdot 0=0$.\\
与假设矛盾,故不存在.
~\\
\end{solution}
~\\
七、估计下列定积分的值.\\

\begin{enumerate}
\item $ \displaystyle\int_0^2 x \mathrm{e}^{-x} \mathrm{~d} x .$\\
\begin{solution}
	$\text { 记 } f(x)=x \cdot e^{-x}, f^{\prime}(x)=(1-x) \cdot e^{-x},\text { 故 } 0<x<1 \text { 时 } f^{\prime}(x)>0;1<x<2 \text { 时 } f^{\prime}(x)<0$.\\
	因此 $f(x)_{\max }=f(1)=\dfrac{1}{e} ,f(x)_{\min }=\min \{f(0) \cdot f(2)\}=0 .$\\
	$\text { 故 } 0 \leqslant \displaystyle\int_0^2 x \cdot e^{-x} d x \leqslant \frac{2}{e}.$
\end{solution}
\item $ \displaystyle\int_{\frac{1}{4}}^{\frac{1}{2}} x^x d x .$ \\
\begin{solution}
	$ \text { 记 } f(x)=x^x=e^{x \cdot \ln x},f^{\prime}(x)=x^x \cdot(1+\ln x). $
	$ \text { 故 } \dfrac{1}{4}<x<\dfrac{1}{e} \text { 时. } f^{\prime}(x)<0,\dfrac{1}{e}<x<\dfrac{1}{2} \text { 时, } f^{\prime}(x)>0. $ \\
	 因此 $f(x)_{\min }=f\left(\dfrac{1}{e}\right)=e^{-\frac{1}{e}}$  ,
	$f(x)_{\max }=\max \left\{f\left(\dfrac{1}{2}\right), f\left(\dfrac{1}{4}\right)\right\}=\dfrac{\sqrt{2}}{2}.$ \\
	$\text { 故 } \dfrac{1}{4} \cdot e^{-\frac{1}{e}} \leqslant \displaystyle\int_{\frac{1}{4}}^{\frac{1}{2}} x^x d x \leqslant \frac{\sqrt{2}}{8}.$
\end{solution}
\end{enumerate}
~\\
九、比较下列各对定积分的大小.\\
\begin{enumerate}
	\item $\displaystyle\int_0^{\frac{\pi}{4}} \sin ^2 x \mathrm{~d} x$ 与 $\displaystyle\int_0^{\frac{\pi}{4}} \sin ^4 x \mathrm{~d} x$;
	\begin{solution}
	当 $x\displaystyle\in\left[0, \frac{\pi}{4}\right]$ 时. $\sin ^2 x \in[0,1)$, 故 $0 \leqslant \sin ^4 x \leqslant \sin ^2 x$.\quad 
	因此:$\displaystyle\int_0^{\frac{\pi}{4}} \sin ^2 x d x \geqslant \int_0^{\frac{\pi}{4}} \sin ^4 x d x$.\\
	\end{solution}

\item $ \displaystyle\int_1^2 \sqrt{5-x} d x \text { 与 } \int_1^2 \sqrt{x+1} d x ;$ 
\begin{solution}
	$\text { 当 } x \displaystyle\in[1,2] \text {时,} \sqrt{5-x} \geqslant \sqrt{3} \text {且 } \sqrt{x+1} \leqslant \sqrt{3} \text {.即 :} \sqrt{5-x} \geqslant \sqrt{x+1} \geqslant 0 .$ 
	所以有:$\displaystyle \int_1^2 \sqrt{5-x} d x \geqslant \int_1^2 \sqrt{x+1} d x.$
\end{solution}
\item $ \displaystyle\int_0^{\frac{\pi}{2}} \mathrm{e}^{-x} \mathrm{~d} \text { 与} \int_0^{\frac{1}{2}} \mathrm{e}^{-\mathrm{sin} x} \text {; }$ 
\begin{solution}
	$ \text { 当 } x \displaystyle\in\left[0, \frac{\pi}{2}\right] \text { 时. } \quad \sin x \leq x, \quad 0 \leq e^{-x} \leq e^{-\sin x}$\quad
	$\text { 故 }\displaystyle \int_0^{\frac{\pi}{2}} e^{-x} d x \leqslant \int_0^{\frac{1}{2}} \mathrm{e}^{-\mathrm{sin} x}.$
\end{solution}

\item $\displaystyle\int_{0}^{\frac{\pi}{2}x}\sin(\sin x) dx$与$\displaystyle\int_{0}^{\frac{\pi}{2}x}\cos(\sin x) dx$.

\begin{solution}\\
	Solution 1.
	注意到 $(\forall x>0) \sin x<x$, 以及 $\cos x$ 在 $[0, \pi / 2]$ 上递减,于是 $\sin \cos x<\cos x<\cos \sin x$. 所以
	$$
	\int_0^{\frac{\pi}{2}} \sin \sin x \mathrm{~d} x=\int_0^{\frac{\pi}{2}} \sin \cos x \mathrm{~d} x<\int_0^{\frac{\pi}{2}} \cos \sin x \mathrm{~d} x.
	$$
	Solution 2.
	$$
	\begin{aligned}
		& \int_0^{\frac{\pi}{2}} \cos \sin x \mathrm{~d} x=\int_0^{\frac{\pi}{2}} \cos \cos x \mathrm{~d} x=\int_0^1 \frac{\cos x}{\sqrt{1-x^2}} \mathrm{~d} x>\int_0^1. \frac{1-\frac{x^2}{2}}{\sqrt{1-x^2}} \mathrm{~d} x=\frac{3 \pi}{8}>1, \\
		& \int_0^{\frac{\pi}{2}} \sin \sin x \mathrm{~d} x=\int_0^1 \frac{\sin x}{\sqrt{1-x^2}} \mathrm{~d} x<\int_0^1 \frac{x}{\sqrt{1-x^2}} \mathrm{~d} x=1.
	\end{aligned}
	$$
		Solution 3 .其实画图也可以,这里不好画,习题课上讲!
\end{solution}
\item $\displaystyle\int_{0}^{\pi}e^{-x^2} dx$与$\displaystyle\int_{\pi}^{2\pi}e^{-x^2}dx  $.
\begin{solution}
	$f(x)=e^{-x^2}$在$R$上单调递减,利用定积分的几何意义易知:$\displaystyle\int_{0}^{\pi}e^{-x^2} dx>\displaystyle\int_{\pi}^{2\pi}e^{-x^2}dx.$
\end{solution}
\end{enumerate}

~\\
十、设 $f(x)$ 和 $g(x)$ 在 $[a, b]$ 上连续, 证明:
\begin{enumerate}
\item 若在 $[a, b]$ 上 $f(x) \geqslant 0$, 且至少有一点 $c \in[a, b]$, 使得 $f(c)>0$, 则 $\displaystyle\int_a^b f(x) \mathrm{d} x>0$; \\
\item 若在 $[a, b]$ 上 $f(x) \geqslant g(x)$, 且至少有一点 $c \in[a, b]$, 使得 $f(c)>g(c)$, 则 $\displaystyle\int_a^b f(x) \mathrm{d} x>\int_a^b g(x) \mathrm{d} x$.\\
	~\\
\end{enumerate}
\begin{proof}
	1. 由连续性. $\exists x=c$ 的小邻域: $[c-\delta, c+\delta]$. S.t. $f(x) \geqslant \dfrac{1}{2} f(c)>0$. 又 在 $x \in[a, b] $上$, f(x) \geqslant 0$.\\
	故$\displaystyle\int_a^b f(x) d x \geqslant \int_{c-\delta}^{c+\delta} f(x) d x \geqslant 2 \delta \cdot \frac{1}{2} f(c)=\delta \cdot \delta(c)>0$. 故得证.$\hfill\square$
\end{proof}
\begin{proof}2.
	设 $F(x)=f(x)-g(x)$. 故 $F(x)$ 连续. 由第1问的证明可得:$\displaystyle\int_a^b F(x) d x>0$. 故
	$\displaystyle\int_a^b f(x) d x>\int_a^b g(x) d x$.$\hfill\square$
\end{proof}

~\\
十一、
\begin{enumerate}
	\item 设 $\displaystyle\int_a^b f(x) \mathrm{d} x=m, \int_c^b f(x) \mathrm{d} x=n$, 则 $\displaystyle\int_c^a f(x) \mathrm{d} x=(C)$.\\
	~\\
	A. $m+n$\qquad
	B. $m-n$\qquad
	C. $n-m$\qquad
	D. 0\\
	
\item 初等函数 $f(x)$ 在其定义区间 $[a, b]$ 上不一定(B).\\
A. 连续\qquad
B. 可导\qquad
C. 存在原函数\qquad
D. 可积\\


\item 下列函数中,在区间 $[-1,3]$ 上不可积的是 (D).\\
A. $f(x)=\left\{\begin{array}{l}3,-1<x<3, \\ 0, x=-1, x=3\end{array}\right.$\quad
B. $f(x)=[x]$\quad
C. $f(x)=\left\{\begin{array}{l}\dfrac{\sin x}{x}, x \neq 0, \\ 1, x=0\end{array}\right.$\quad
D. $f(x)=\left\{\begin{array}{l}\mathrm{e}^{\frac{1}{2}}, x \neq 0, \\ 1, x=0\end{array}\right.$\\
\end{enumerate}

~\\
十二、求数列极限: $\displaystyle\lim\limits _{n \rightarrow \infty} \int_n^{n+\rho} \dfrac{x^2}{x^2+a^2} \mathrm{~d} x$\\


\begin{solution}
	原式 $=\lim\limits _{n \rightarrow \infty} p \cdot \dfrac{\xi^2}{\xi^2+a^2} \quad(n \leqslant \xi \leqslant n+p)=\lim\limits _{\xi \rightarrow \infty} p \cdot \dfrac{1}{1+\dfrac{a^2}{\xi^2}}=p.$ \\
	~\\
\end{solution}

~\\
十三、设函数 $f(x)$ 连续,求 $\lim\limits _{h \rightarrow 0} \dfrac{1}{h} \displaystyle\int_a^{a+h} f(x) \mathrm{d} x$.\\
\begin{solution}
	 原式$=\lim\limits _{h \rightarrow 0} \dfrac{1}{h} \displaystyle\int_a^{a+h} f(x) d x $
	$=\lim\limits_{h \rightarrow 0} \dfrac{1}{h} \cdot hf(\xi).\quad(a \leqslant \xi \leqslant a+h) $
	$=\lim\limits _{\xi\rightarrow a} f(\xi)=  f(a).$
\end{solution}

\end{document}
